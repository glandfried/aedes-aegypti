\section{aedes-aegypti}\label{aedes-aegypti}

El Aedes aegypti es un mosquito que transmite más de 100 enfermedades.
La población del Aedes aegypti se instaló en la Argentina a mediados de
la década del 90 del siglo. Las estrategias para evitar la
proliferación, basadas en el fumigación, han fracasado rotundamente.
Desde entonces su población no ha dejado de crecer, y con ella la
frecuencia de las epidemias relacionados al mosquito.

La Ciudad de Buenos Aires tiene un convenio desde 1998 con el Grupo de
Estudios de Mosquitos de la Facultad de Ciencias Exactas y Naturales de
la Universidad de Buenos Aires, a cargo de Nicolas Schweigmann.
\textbf{Tienen disponible mapas, semana a semana, de la población de
mosquitos} en la Ciudad de Buenos Aires desde 1998 hasta la fecha. Solo
en la Ciudad de Buenos Aires la población de mosquitos se triplicó.

Según Nicolás Schweigmann, el fracaso de es consecuencia de la
implementación de políticas basadas en el paradigma químico de control
de la población por fumigación, y nos propone una pensar el problema
desde el \textbf{paradigma ambiental}. Este tipo de mosquito no
prolifera en zonas verdes, la proliferación de mosquitos depende de los
hábitos socio-culturales de la población urbana. Lo que necesitamos es
adquirir, como población, los hábitos que hagan saludable a nuestros
ambientes. Lo interesante es que, gracias a que el mosquito vive a no
más de 40 metros de sus criaderos, \textbf{la unidad
ambiental-comunitaria es ``la manzana''}. Y esto permite que acciones
locales bien implementadas logren convertirse en \textbf{zonas libres de
Aedes aegypti}.

El protocolo de entrar a las casas para descacharrear no sirve mientras
no esté permitido entrar a todas las casas. Y aunque estuviera
permitido, la estructura estatal sería incapaz de mantener todas las
viviendas libres criaderos sin la colaboración activa de la población.
La \textbf{solución requiere del aprendizaje social} y el cambio de
hábitos de la población. Es fundamental implementar un programa a largo
plazo que invite activamente a la población a actuar durante todo el año
en en su propia unidad ambiental-comunitaria.

Hay que prestar atención en \textbf{los niños y niñas como agentes de
control sanitario}. A temprana edad exploran, y son los que mejor
conocen las casas, mejor que los adultos que allí habitan y más
efectivos en encontrar criaderos que cualquier agente estatal, por más
experiencia que tenga este. El Estado Provincial debería pensar en
incluir en la currícula escolar la enseñanza el ciclo de vida del
mosquito Aedes y las estrategias para regularlo. Al menos deberia dar el
ejemplo, garantizando que todos sus predios públicos, \textbf{escuelas y
hospitales, sean zonas libres de Aedes aegypti}. Esto ya sería un éxito
inédito para mostrar y replicar.

\textbf{Nicolás Schweigmann ofrece} 1. Ser la cara que le explique a la
población que es necesario \textbf{descacharrear en invierno}. 2. La
exactas-UBA para entrenar empleados públicos y/o trabajadores de la
economía popular. 3. A establecer una colaboración con su grupo,
runiones regulares durante todo el año, donde evaluar las acciones a
implementar en la Provincia de Buenos Aires.

\textbf{Materiales}: 1. Plan Maestro diseñado por Nicolás Schweigmann
para CABA 2.
\href{http://www.ege.fcen.uba.ar/novedades/extension-y-popularizacion/reflexiones-ambientales/}{Reflexiones
ambientales urbanas}, 48 documentos cortos de 1 página. 3. Datos
generados por el convenio de mosquitos de CABA.
